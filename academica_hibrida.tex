\documentclass{article}
\usepackage[utf8]{inputenc}
\usepackage{graphicx}

\title{[RC] CDN}
\author{afonsojoao96 }
\date{October 2015}

\begin{document}

\section{Academic Architecure (peer-to-peer)}
As said before \emph{CDN Commercial Architecture} brings mostly advantages to
clients but it also comes with it's disadvantages that still
need to be improve. 

The main disadvantage of this architecture is the high financial cost. This
happens due to the large distributed infrastructure that causes expensive
financial cost of server renting. Therefore only big companies like \emph{Akamai}
(leading company in this area) can afford it, making de \texttt{CDN} market very
centralized and consequently making the fees associated with the service
really high. As a result, small and medium sized Internet content providers
are not allowed to grow because they are not in a financially viable position
to rent the services of commercial \texttt{CDN} providers. Only large firms can
afford it.


In order to improve this big disadvantage another architecture was implemented:
the \emph{Academic Architecture}.

\subsection{Âmbitos de aplicação}

Contrary to the commercial architectures, \emph{academic} ones are about
non-profit peer-to-peer architectures, where each user shares a  little of the
machine's resources (memmory, CPU cicles, etc...) with the remaining
(intervenientes). In other words, this \emph{peer-to-peer} architecture
creates a distributed storage medium that allows the search, the publishing
and recovery of files by members of the network.

\subsubsection{Principais vantagens}

This alternative makes the network independent from ??entrepreneurial?? servers,
which makes it singnificantly cheaper (the price to pay is the small shared resources).

Another advantage of this architecture comes from the increased quality and speed
of the network in proportion to the number of users, i. e., the more users it has,
the faster and the more reliable it becomes. 
For instance, high quality streaming to a great number of users may benefit from
using an \emph{academic architecture}, as the provider of that same stream won't
have to deal with the cost of renting a server and, beyond that, the high number
of users will guarantee a wide array of available resources to a global video
transmission.  

\subsection{Desafios associados}
Despite being better,when it comes to costs,
than the previous option, it still has some disadvantages.
The biggest disadvantage is that availability and quality of the content
reception depends on content providers that are volunteers and therefore don't
follow a clear set of rules causing a lack of standard specification of
\texttt{CDN} interfaces.
Consequently, this makes it hard to maintenance and support end users.

\newpage



%%%%%%%%%%%%%%%%%%%%%%%%%%%%%%
\section{Hybrid Architecture}
%%%%%%%%%%%%%%%%%%%%%%%%%%%%%%


As it has been shown, both \emph{comercial} architecture and 
\emph{academic} architecture present some drawbacks that may 
??make unviable?? it's utilization in medium scale. If on one hand,
\emph{commercial} architecture has it's cost only viable for 
large scale aplications, on the other hand the \emph{academic}
one is dependant on the number and quality of it's \emph{peers}
which can cause some instability and lack of sturdy.
In this way, it becomes central to look for more reliable alternatives
that can combine \emph{the best of both worlds}. A currently already
proposed architecture is the ``Distributed Content Delivery Networks''
\texttt{DCDN}, which is an hybrid model that is between 
\emph{comercials} and \emph{academics} \texttt{CDN}.




\subsection{Âmbitos de Aplicação}

\subsubsection{Framework}
The \texttt{DCDN} are composed by a well defined hierarchical structure
that allows a better relationship among the network's elements.
This way, it is possible to defined to following entities (from higher
to the lower level of the hierarchy):

% Talvez fosse importante pôr a imagem que está no paper,
% dá para perceber muito melhor a hierarquia e a distribuição
% de conteúdo nem que fosse para a apresentação

\begin{itemize}
\item \textbf{Content provider} - creator and manager of the shared content;

\item \textbf{Administrators} - a group of users with privileges on the 
network responsible for the network's maintenance and support

\item \textbf{Servers} - a set of servers, that are utilized, not to
data storage, but for the efficient distribution of network content.
They can be \emph{Master} tipe (directly connected to the
\emph{content provider}, distributing the content by the various regions
where the \texttt{CDN} acts) or \emph{Local} (connected to the 
\emph{master servers} and  channel the content to the \emph{surrogates}
acting on a closer level to the client).

\item \textbf{Surrogates} - set of users of P2P network whose
resources are shared;

\item \textbf{Client} - final entity, that gets the content;

\end{itemize}



\subsubsection{Método de Distribuição do Conteúdo}

Given that the main point of \texttt{DCDN} is to optimize the content acess
and minimize it's costs, it makes sense that the content replicas
are as close as possible to the client, i.e., on the \emph{surrogates}.
This way, content distribution is made sequentially in the following way:
\begin{enumerate}

\item The \emph{content providers} ask permission to 
\emph{administrators} to insert a new content on the network;

\item If the request is successful, \emph{master} servers send
the updated content to the \emph{local} servers and from them to 
the \emph{surrogates};

\item At the same time that the content is sent, servers update their
records with the same contents being that the \emph{master} will own more general data
(like the network area where the content is distributed) 
and the \emph{local} more specific informations (i.e, that \emph{surrogates}
are in possession of the contents);

\item When there is an content access request by the 
\emph{client}, the \emph{local} server will choose the valid \emph{surrogates}
for the content share. This way, this is distributed between the
\emph{surrogates} and the \emph{client} through the \emph{peer-to-peer} network.
\end{enumerate}



\subsubsection{Principais Vantagens}

The main advantage of this architecture, when compared to the \emph{comercial},
is it's reduced cost of implementation (because there is a need 
of much less servers).

When compared to the \emph{academic} architecture, this model assures a
better content stability and availability, for it does not depend 
only on the volunteer \emph{peer-to-peer} effort. 

With this architecture becomes also possible to restrict the content
with the client's location, because the content distribution is made
by the \emph{master} and \emph{local} servers. This way, if the 
\emph{content providers} intend that a specific content will only be
accessible from determined place, the content will only be sent to
that region's servers, therefore preventing other region's users
to access it.


\subsection{Desafios Associados}

This architecture has some drawbacks, the main being that this is
only an hypothetical model, with few or none real implementation
and, consequently is a little tested platform.

Relatively to the \emph{academic} and \emph{comercial} \texttt{CDN},
the \texttt{DCDN} adds some levels of complexity to the structure and
hierarchy, wich may difficult it's physical implementation.

Once more, the lack of standards to the \texttt{CDN} is also
a drawback that must be deal with as quicly as possible.


\end{document}
