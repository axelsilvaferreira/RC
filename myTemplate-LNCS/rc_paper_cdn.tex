\documentclass{llncs}
\usepackage{times}
\usepackage[T1]{fontenc}

% Comentar para not MAC Users
\usepackage[applemac]{inputenc}

\usepackage{a4}
%\usepackage[margin=3cm,nohead]{geometry}
\usepackage{epstopdf}
\usepackage{graphicx}
\usepackage{fancyvrb}
\usepackage{amsmath}
%\renewcommand{\baselinestretch}{1.5}

% CAPITAL LETTER LETTRINE
\usepackage{lettrine}

\begin{document}
\mainmatter
\title{Overview of Content-Delivery Networks Architecture}

\titlerunning{Paper Title}

\author{Afonso Silva\and Alfredo Gomes \and Axel Ferreira}

\authorrunning{Autor1 \and Autor2 \and Autor3}

\institute{University of Minho, Department of Informatics, 4710-057 Braga, Portugal\\
e-mail: \{a70387,a71655,a53064\}@alunos.uminho.pt
}

\date{\today}

\bibliographystyle{splncs}
%---------------- TITLE
\maketitle

%---------------- TABLE OF CONTENTS
%\tableofcontents
%\newpage

%%%%%%%%%%%%%%%%%%%%%%%%%%%%%%%%%%%%%%%%%%%%%%%%%%%%%%%%
%---------------- ABSTRACT							%()
%%%%%%%%%%%%%%%%%%%%%%%%%%%%%%%%%%%%%%%%%%%%%%%%%%%%%%%%
\begin{abstract}
Resumo...
\end{abstract}
%----------------
%%%%%%%%%%%%%%%%%%%%%%%%%%%%%%%%%%%%%%%%%%%%%%%%%%%%%%%%
\section{Introduction}									%(Axel)
%%%%%%%%%%%%%%%%%%%%%%%%%%%%%%%%%%%%%%%%%%%%%%%%%%%%%%%%

%\subsection{Context}
\lettrine[lines=2]{T}{he} Internet has its origins in the early 1980's and presented an exponential growth when commercial companies started to link to the existing academic and military networks during the 90's. As the popularity of the Internet increased the number of devices connected started to see an exponential growth.\\ 
% \subsection{Problems }
At that time (analogue) networks were unreliable and internet communication protocols were designed in a robust fashion. As an example, Hypertext Transfer Protocol (HTTP) was designed to survive multiple packet losses and thus being a very chatty protocol, there are multiple RTTs (Round-trip Time) this causes a latency problem over long distance communication.
Modern (digital) networks are faster and more reliable, but most of the core protocols mentioned above don't take advantage of the increased reliability.\\
% @@@@ DUE TO Backwards compatibility ?�@@@@
As the internet role in daily life grew, new technologies star to emerge providing images videos and other dynamic content. This caused a bottleneck in servers with popular content.
The above problem led to the development of a new solution called Content Delivery Network.\\
%\subsection{What is a CDN?}
A Content Delivery Network (CDN) is a large network of distributed storage servers which cash the content in multiple locations widely spread. 

This reduces packet losses, latency and total amount of traffic in the network as shown in the diagram. It also provides redundancy.
%\subsection{How does it Work?}

\subsection{Purpose} %Prop�sito do paper
	1 - overview das 3 arquitecturas principais
		Objetivo do paper - num �nico texto (organizar informa��o)
		(No fim da intro)
			- impacto das CDN na sociedade (�mbitos de aplica��o)
			- dado o impacto social das CDN => � importante reduzir os custos associados �s CDN (infraestruturas) de forma a permitir que um numero maior de empresas (de pequena e media dimens�o) ou particulares possam usufruir delas. =>impacto Redu��o da carga nas internet

	



%------ MULTI LINE COMENT -------
\if false
%Ideias: 	
%	- Created in the second half of the 90's
%	- Fast internet popularity growth rate
%	- Created to respond to the following problems
%		- TCP slow start
%		- HTTP Chatty Protocol
%			->Very robust, works even with multiple packet losses
%			    But not optimised for current networks and thus not taking advantage
%			    of more modern better-quality networks
%		- Multiple (30-50) RTTs / page
%		- Connection load at origin
%		- Content Delivery
\fi 
%%%%%%%%%%%%%%%%%%%%%%%%%%%%%%%%%%%%%%%%%%%%%%%%%%%%%%%%
\section{�mbito de aplica��o}								%(Afonso)
%%%%%%%%%%%%%%%%%%%%%%%%%%%%%%%%%%%%%%%%%%%%%%%%%%%%%%%%

Netflix FHD 5Mbit/seg
	   UHD 25Mbit/seg

What are the Advantages?
	% - Low latency - Content cashed close to consumer
	% - Redundancy - In case of natural disaster (ex.: incendio uMinho)
	% - Safer - Against (DoS) Attacks & Protects origin.
	% - Cheap to get going (pay as you go) no need for big investment at beguinning
	% - Capacity on Demand - Dominos friday noght





According to Table~\ref{tab:TabelaExemplo}...


%%%%%%%%%%%%%%%%%%%%%%%%%%%%%%%%%%%%%%%%%%%%%%%%%%%%%%%%
\section{Desafios associados}								%(Alfredo)
%%%%%%%%%%%%%%%%%%%%%%%%%%%%%%%%%%%%%%%%%%%%%%%%%%%%%%%%





\if
	-Onde Colocar os servidores da CDN e Porqu�?
	-O paper "Standard IPTV CDN Architecture Interfaces" Fala sobre desafios
		� Falta de standards
		� Incompatibilidade entre as CDNs existentes (impossibilita a mobilidade)
		
	{What are the Disadvantages?}
	% Provider End (CDN Proviedr)
	% Client End (Content Provider)
	%-Usually expensive (due to large distributed infrastructure (high mainenece costs))
	%-Lack of standards and Norms => hard to change provider
\fi
%%%%%%%%%%%%%%%%%%%%%%%%%%%%%%%%%%%%%%%%%%%%%%%%%%%%%%%%
\section{Arquiteturas relevantes na �rea}						%(Axel)
%%%%%%%%%%%%%%%%%%%%%%%%%%%%%%%%%%%%%%%%%%%%%%%%%%%%%%%%






\if
	Arquiteturas Propostas
		CDN Comercial (Uses CDN Servers at ISPs to provide content)
		CDN P2P (Uses Both Dedicated Servers and Peer-user-owned Computers)
		(Hybrid) DCDN [Distributed CDN]
		-justificacao- baixa de precos e acessivel a + gente
\fi 

%%%%%%%%%%%%%%%%%%%%%%%%%%%%%%%%%%%%%%%%%%%%%%%%%%%%%%%%
\section{Lista de projetos atuais}							%()
%%%%%%%%%%%%%%%%%%%%%%%%%%%%%%%%%%%%%%%%%%%%%%%%%%%%%%%%	






	
\if
	Cloud Content Delivery Networks
	
		Free CDNs
			BootStrap CDN\\
			CloudFlare\\
			CCDN\\
			Incapsula\\
			\dots\\
		Telco CDN\\
			AT\&T\\
			Verizon\\
			\dots\\
		Traditional CDNs\\
			Akamai\\
				Amazon\\
			Windows Azure\\
			HP Cloud\\
			CloudFlair\\
			\dots 

		Commercial CDN P2P\\
			BitTorrent\\
			Internap\\
			\dots\\
\fi


\section{Conclusion and Future Work}




%%%%%%%%%%%%%%%%%%%%%%%%%%%%%%%%%%%%%%%%%%%%%%%%%%%%%%%% 
%--------------- G L O S S A R I O ----------------%
%%%%%%%%%%%%%%%%%%%%%%%%%%%%%%%%%%%%%%%%%%%%%%%%%%%%%%%%
	CDN - Content-delivery Network
	DCDN - Distributed Content-delivery Network
	DNS - Domain Name System
	RTT - round-trip time
	PoP - Points of Presence
	ASP - Application Service Provider
	SaaS - Software as a Service
	Caching -
	Mirroring -
	
%%%%%%%%%%%%%%%%%%%%%%%%%%%%%%%%%%%%%%%%%%%%%%%%%%%%%%%%
%----------- B I B L I O G R A F I A -------------%
%%%%%%%%%%%%%%%%%%%%%%%%%%%%%%%%%%%%%%%%%%%%%%%%%%%%%%%%
%UNCOMMENT para a bibliografia 
%% ficheirodebibliografia.bib
%\bibliography{ficheirodebibliografia}

% OU

% Inserir directamente os v�rios \bibitem 
\begin{thebibliography}{1}
\bibitem{Zadeh65}
Zadeh, L.:
\newblock {Fuzzy sets} (1965)

\bibitem{Nguyen99}
Nguyen, H., Walker, E.:
\newblock {First course in fuzzy logic}.
\newblock {Boca Raton: Chapman and Hall/CRC Press} (1999)
\end{thebibliography}



\end{document}