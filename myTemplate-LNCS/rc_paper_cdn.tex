\documentclass{llncs}
\usepackage{times}
\usepackage[T1]{fontenc}

% Comentar para not MAC Users
\usepackage[applemac]{inputenc}

\usepackage{a4}
%\usepackage[margin=3cm,nohead]{geometry}
\usepackage{epstopdf}
\usepackage{graphicx}
\usepackage{fancyvrb}
\usepackage{amsmath}
%\renewcommand{\baselinestretch}{1.5}

\begin{document}
\mainmatter
\title{T10 - [\dots in] Content-Delivery Networks}

\titlerunning{Paper Title}

\author{Afonso Silva\and Alfredo Gomes \and Axel Ferreira}

\authorrunning{Autor1 \and Autor2 \and Autor3}

\institute{
University of Minho, Department of Informatics, 4710-057 Braga, Portugal\\
e-mail: \{a70387,a71655,a53064\}@alunos.uminho.pt
}

\date{\today}

\bibliographystyle{splncs}
%---------------- TITLE
\maketitle

%---------------- TABLE OF CONTENTS
\tableofcontents
%\newpage
%---------------- ABSTRACT
\begin{abstract}
Resumo...
\end{abstract}
%----------------
\section{Introduction}

%Contextualiza��o
The  Internet has it's origins in the early 1980's. And started to have an exponential growth when commercial companies started linking to the existing academic and military networks during the 90's. As the popularity if the Internet increased the number of devices connected started to see an exponential growth.\\
% Problemas que levaram a cria��o
At the time (analogue) networks were unreliable and thus internet communication protocols were designed in a robust fashion. There are multiple RTTs (Round-trip Time) (30-50 per page). TCP is a reliable but slow starting protocol. Hypertext Transfer Protocol (HTTP) was designed to survive multiple packet losses and thus being very chatty, this causes a latency problem over long distance communication.
Modern (digital) networks are faster and more reliable, but most of the core protocols above don't take advantage of the increased reliability.\\ @@@@@@@@@@@(DUE TO Backwards compatibility ?????? )@@@@@@@@@@@@@\\
Another problem introduced with the ever growing popularity of the internet was caused as connection speeds started to grow when content increased in size. New technologies star to provide images, videos and other dynamic content. This caused a bottleneck problem at the origin (content providers).
With the continuous Internet growth a response was necessary to solve the above problems and Content Delivery Networks started to emerge (in 95's)..


%------ MULTI LINE COMENT -------
\if false
Ideias: 	
	- Created in the second half of the 90's
	- Fast internet popularity grouth rate
	- Created to respond to the following problems
		- TCP slow start
		- HTTP Chatty Protocol
			->Very robust, works even with multiple packet losses
			    But not optimised for current networks and thus not taking advantage
			    of more modern better-quality networks
		- Multiple (30-50) RTTs / page
		- Connection load at origin
		- Content Delivery
\fi 


\section{�mbito de aplica��o}

Netflix FHD 5Mbit/seg
	   UHD 25Mbit/seg


%UNCOMMENT se necess�rio
%De acordo com o ilustrado na Figura~\ref{fig:controller}
% Exemplo para inser��o de uma figura
%\begin{figure}
%\begin{center}
%\includegraphics[scale=0.40]{figura.pdf} 
%\end{center}
%\caption{\label{fig:controller}Architecture of the unified QoS metric fuzzy controller.}
%\end{figure} 

According to Table~\ref{tab:TabelaExemplo}...

% Exemplo de uma tabela com duas colunas
\begin{figure}
\centering
\begin{tabular}{|c|c|}\hline
(a) Delay and jiiter & (b) Delay and loss \\ \hline

(c) Delay and throughput & (d) Jitter and loss \\ \hline

(e) Jitter and throughput & (f) Loss and throughput \\ \hline
\end{tabular}
\caption{\label{tab:TabelaExemplo}Tabela exemplo.}
\end{figure}

%\section{Simulation Scenario}
\section{Desafios associados}

\section{Propostas relevantes na �rea}
\if
	CDN Comercial (Uses CDN Servers at ISPs to provide content)
	CDN P2P (Uses Both Dedicated Servers and Peer-user-owned Computers)
	DCDN [Distributed CDN]
\fi

\section{Lista de projetos atuais}

\section{Conclusion and Future Work}




 
%--------------- G L O S S A R I O ----------------%
	CDN - Content-delivery Network
	DCDN - Distributed Content-delivery Network
	DNS - Domain Name System
	RTT - round-trip time
	PoP - Points of Presence
	ASP - Application Service Provider
	SaaS - Software as a Service
	Caching -
	Mirroring -
	

%----------- B I B L I O G R A F I A -------------%
%UNCOMMENT para a bibliografia 
%% ficheirodebibliografia.bib
%\bibliography{ficheirodebibliografia}

% OU

% Inserir directamente os v�rios \bibitem 
\begin{thebibliography}{1}
\bibitem{Zadeh65}
Zadeh, L.:
\newblock {Fuzzy sets} (1965)

\bibitem{Nguyen99}
Nguyen, H., Walker, E.:
\newblock {First course in fuzzy logic}.
\newblock {Boca Raton: Chapman and Hall/CRC Press} (1999)
\end{thebibliography}



\end{document}